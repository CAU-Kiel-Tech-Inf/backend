\documentclass[12pt,a4paper, ngerman, oneside]{scrartcl}
\usepackage[ngerman]{babel}
\usepackage[utf8]{inputenc}
%\usepackage[T1]{fontenc}
%\usepackage{lmodern}
\usepackage{ucs}
\usepackage{amsmath}
\usepackage{amsfonts}
\usepackage{amssymb}
\usepackage{graphics}
\usepackage{paralist}
\usepackage[linkcolor=black]{hyperref}
\usepackage{listings} \lstset{numbers=left, numberstyle=\tiny, numbersep=5pt} \lstset{language=Java}
%% Grafiken
\usepackage[pdftex]{graphicx}
\usepackage{epsfig} 
\hypersetup{% 
  colorlinks=true, 
}
\sloppy
\hyphenpenalty=100000

\date{Software-Challenge Germany 2013\\Stand \today}


%\author{Niklas Paulsen, npau@informatik.uni-kiel.de }
\title{Client-Server Kommunikation Cartagena}

\begin{document}
\maketitle
\thispagestyle{empty}
\tableofcontents
\thispagestyle{empty}
\newpage
\setcounter{page}{1}
\section{Einleitung}
\subsection*{Zum Inhalt} Wie in den letzten Jahren wird zur Client-Server
Kommunikation ein XML-Protokoll genutzt\footnote{siehe auch:
\htmladdnormallink{SC Wiki: Die Schnittstelle zum
Server}{http://sc-doku.gfxpro.eu/wiki/Die_Schnittstelle_zum_Server}}. In diesem
Dokument wird die Kommunikationsschnittstelle definiert, sodass ein komplett
eigener Client geschrieben werden kann. Es wird hier nicht die vollständige
Kommunikation dokumentiert bzw. definiert, dennoch alles, womit ein Client
umgehen können muss, um spielfähig zu sein.
\subsection*{An wen richtet sich dieses Dokument?} Die Kommunikation mit dem
Spielserver ist für diejenigen, die aufbauend auf dem Simpleclient
programmieren, unwichtig. Dort steht bereits ein funktionierender Client bereit
und es muss nur die Spiellogik entworfen werden. \\
Nur wer einen komplett eigenen Client entwerfen will, beispielsweise um die
Programmiersprache frei wählen zu können, benötigt die Definitionen.

\subsection*{Hinweise} Falls Sie beabsichtigen sollten, diese
Kommunikationsschnittstelle zu realisieren, sei darauf hingewiesen, dass es im
Verlauf des Wettbewerbes möglich ist, dass weitere, hier noch nicht aufgeführte
Elemente zur Kommunikationsschnittstelle hinzugefügt werden. Um auch bei solchen
Änderungen sicher zu sein, dass ihr Client fehlerfrei mit dem Server
kommunizieren kann, empfehlen wir Ihnen, beim Auslesen des XML jegliche Daten zu
verwerfen, die hier nicht weiter definiert sind. \bigskip \\
Die vom Institut bereitgestellten Programme (Server, Simpleclient) nutzen eine
Bibliothek um Java-Objekte direkt in XML zu konvertieren und umgekehrt. 
Dabei werden XML-Nachrichten nicht mit einem newline abgeschlossen.
%Dabei
%werden den XML-Tags jeweils \textit{id}-Attribute beigefügt. Auch diese können
%vernachlässigt werden.

\subsection*{Wie das Dokument zu lesen ist}
Es finden sich für einzelne Arten von Nachrichten kurze Definitionen. Dabei ist ein allgemeiner XML-Code gegeben, bei dem Attributwerte durch Variablen ersetzt sind, die über dem Code kurz erläutert werden.\\
Eingebettete XML-Elemente werden hier allgemein als \begin{verbatim}
{TAGNAME}
\end{verbatim} geschrieben, wenn an der Stelle eine beliebige Anzahl von \textit{TAGNAME}-Tags eingebettet sein \textit{kann} und als \begin{verbatim}
[TAGNAME]
\end{verbatim}, wenn an der Stelle ein \textit{TAGNAME}-Tag optional stehen kann. \\

\subsection{Beispiel-Definition}
Die Definitionen von a- und b-Tags unten lassen unter anderem folgende a-Tags zu:
\begin{verbatim}
- <a name="Sinnvoll">
  <\a>
- <a name="Sinnvoll">
      <b anzahl="2"/>
  <\a>
- <a name="Hans">
      <b anzahl="2"/>
      <b anzahl="2"/>
      <b anzahl="2"/>
  </a>
\end{verbatim}
\subsubsection*{Definition 'b'}
\begin{verbatim}
<b anzahl="2">
\end{verbatim}
\subsubsection*{Definition 'a'}
\begin{description}
\item[N] Ein beliebiger Name
\item[b] ein b-Tag wie in obiger Definition
\end{description}
\begin{verbatim}
<a name="N" >
	{b}
<\a>
\end{verbatim}

\newpage
\part{Client $\rightarrow$ Server}
\section{Spiel betreten}
\subsection{Ohne Reservierung}
Betritt ein beliebiges offenes Spiel:
\begin{verbatim}
<join gameType="swc_2013_cartagena"/>
\end{verbatim}
\subsection{Mit Reservierungscode}
Ist ein Reservierungscode gegeben, so kann man den durch den Code gegebenen Platz betreten.
\begin{description}
\item[RC] Reservierungscode
\end{description}
\begin{verbatim}
<joinPrepared reservationCode="RC"/>
\end{verbatim}

\section{Züge senden}

\subsection{\label{moveContainer}Der MoveContainer}
Der MoveContainer kann bis zu 3 Teilzüge beherbergen. Sollen weniger als 3
Teilzüge mitgeschickt werden, können third-, second- oder auch firstMove
weggelassen werden.
\begin{description}
\item[RID] ID des Raumes
\item[ZUG] je nach Art des Zuges Informationen über diesen. (siehe
Abschnitt~\ref{forwardMove} und ~\ref{backwardMove} )
\end{description}
\begin{verbatim}
<room roomId="RID">
  <data class="moveContainer">
    <firstMove ZUG/>
    <secondMove ZUG/>
    <thirdMove ZUG/>
  </data>
</room>

\end{verbatim}
\subsection{Vorwärtszug}
\label{forwardMove}
\begin{description}
\item[ZUGNUMMER] firstMove, secondMove oder thirdMove
\item[FID] der Index des Startfeldes
\item[SYMBOL] das Symbol der Karte, welche abgelegt werden soll.(siehe
Abschnitt \ref{symbols})
\end{description}
\begin{verbatim}
	<ZUGNUMMER class="forwardMove" fieldIndex="FID" symbol="SYMBOL"/>
\end{verbatim}

\subsection{Rückwärtszug}
\label{backwardMove}
\begin{description}
\item[ZUGNUMMER] firstMove, secondMove oder thirdMove
\item[FID] der Index des Startfeldes
\end{description}
\begin{verbatim}
	<ZUGNUMMER class="backwardMove" fieldIndex="FID"/>
\end{verbatim}


\subsection{\label{symbols}Symbole}

Die Strings der Symbolnamen sind:
\begin{compactenum}
\item SKULL
\item HAT
\item DAGGER
\item BOTTLE
\item KEY
\item PISTOL
\end{compactenum}
Die Erkennung der Strings ist case-sensitive, alle Buchstaben müssen groß
sein.
\subsection{Debughints}
Zügen können Debug-Informationen beigefügt werden:
\begin{description}
\item[S] Informationen zum Zug als String
\end{description}
\begin{verbatim}
 <hint content="S"/>
\end{verbatim}
Damit sieht beispielsweise ein MoveContainer mit zwei Teilzügen so aus:
% \begin{verbatim}
% <room roomId="RID">
%   <data class="manhattan:build" city="C" slot="P" size="S">
%   	<hint content="höchstes Gebäude sichern"/>
%   	<hint content="..."/>
%   </data>
% </room>
% \end{verbatim}
\begin{verbatim}
<room roomId="RID">
  <data class="moveContainer">
    <firstMove class="backwardMove" fieldIndex="FID">
     	<hint content="2 Karten ziehen...."/>
     	<hint content="Noch ein Hint"/>
    </firstMove>
    <secondMove class="forwardMove" fieldIndex="FID" symbol="SYMBOL">
    	<hint content="Piraten ins Zielfeld bewegen..."/>
    </secondMove>
  </data>
</room>
\end{verbatim}



\newpage
\part{Server $\rightarrow$ Client}
\begin{description}
\item[RID] ID des Raumes, in dem das Spiel stattfindet
\end{description}

\section{Raum beigetreten}
 \begin{verbatim}
 <joined roomId="RID"/>
 \end{verbatim}

\section{Willkommensnachricht}
Der Spieler erhält zu Spielbeginn eine Willkommensnachricht, die ihm seine Farbe mitteilt.
\begin{description}
\item[C] Spielerfarbe (RED/BLUE)
\end{description}
\begin{verbatim}
<room roomId="RID">
  <data class="welcome" color="C"/>
</room>
\end{verbatim}

\section{Spielstatus}
Es folgt die Beschreibung des Spielstatus, der vor jeder Zugaufforderung an die Clients gesendet wird. Das Spielstatus-Tag ist dabei noch in einem \textit{data}-Tag der Klasse \textit{memento} gewrappt:
\subsection{memento}
\begin{description}
\item[status] wie in \ref{state} definiert
\end{description}
\begin{verbatim}
<room roomId="RID"> 
  <data class="memento"> 
  	status
  </data> 
</room>
\end{verbatim}

\subsection{\label{state}Status}
\begin{description}
\item[Z] aktuelle Zugzahl
\item[C] Spieler, der an der Reihe ist (RED/BLUE)
\item[red, blue] wie in \ref{player} definiert
\item[card] wie in \ref{card} definiert. Die Karte, welche als Nächste gezogen
werden kann, kommt zuerst.
\item[lastMove] Letzter getätigter Zug (nicht in der ersten Runde), wie in
\ref{lastmove} definiert
\item[board] Das Spielbrett, wie in \ref{board} definiert
\item[condition] Spielergebnis, wie in \ref{gameend} definiert; nur zum Spielende
\end{description}
\begin{verbatim}
<state class="state" turn="Z" currentPlayer="C">
  red
  blue
  <openCards>
    {card}
  </openCards>
  [lastMove]
  [board]
  [condition]
</state>

\end{verbatim}

\subsection{\label{player}Spieler}
\begin{description}
\item[C] Farbe (red/blue)
\item[N] Anzeigename
\item[COL] Farbe ``RED''/''BLUE''
\item[P] Punktekonto
\item[card] Spielkarte, wie in \ref{card} definiert
\end{description}
\begin{verbatim}
<C displayName="N" color="COL" points="P">
  <cards>
    {card}
  </cards>
</C>
\end{verbatim}


\subsection{\label{card}Karte}
\begin{description}
\item[S] Symbol der Karte wie in \ref{symbols} definiert.
\end{description}
\begin{verbatim} <card symbol="S"/>
\end{verbatim}

\subsection{\label{board}Spielbrett}
\begin{description}
\item[FIELD] Ein Spielfeld wie in \ref{sec:field} definiert.
\end{description}
\begin{verbatim}
<board>
 <fields>
 	{FIELD}
 </fields>
</board>
\end{verbatim}

\subsection{\label{sec:field}Spielfeld}
\subsubsection{\label{sec:startEndField} Start- und Zielfeld}
\begin{description}
\item[T] Typ des Feldes START/FINISH
\item[PIRATE] Pirat, wie in \ref{sec:pirate} definiert.
\end{description}
\begin{verbatim}
<field type="T">
  <pirates>
  	{PIRATE}
  </pirates>
</field>
\end{verbatim}

\subsubsection{\label{sec:symbolField} Symbolfeld}
\begin{description}
\item[S] Symbol, welches das Feld trägt, siehe \ref{symbols}
\item[PIRATE] Pirat, wie in \ref{sec:pirate} definiert.
\end{description}
\begin{verbatim}
<field type="SYMBOL" symbol="S">
  <pirates>
  	{PIRATE}
  </pirates>
</field>
\end{verbatim}

\subsection{\label{sec:pirate}Pirat}
\begin{description}
\item[O] Besitzer RED/BLUE
\end{description}
\begin{verbatim}
<pirate owner="O"/>
\end{verbatim}



\subsection{\label{lastmove}Letzter Zug}
Der letzte Zug ist ein MoveContainer (siehe hierzu \ref{moveContainer}).
\begin{description}
\item[ZUG] Informationen über den Zug. (siehe \ref{forwardMove} und
\ref{backwardMove})
\end{description}
\begin{verbatim}
<lastMove>
  <firstMove ZUG>
  <secondMove ZUG>
  <thirdMove ZUG>
</lastMove>
\end{verbatim}

\section{\label{moverequest}Zug-Anforderung}
Eine simple Nachricht fordert zum Zug auf:
\begin{verbatim}
<room roomId="RID">
  <data class="sc.framework.plugins.protocol.MoveRequest"/>
</room>
\end{verbatim}

\section{Fehler}
Ein \glqq spielfähiger\grqq\ Client muss nicht mit Fehlern umgehen können.
Fehlerhafte Züge beispielsweise resultieren in einem vorzeitigen Ende des
Spieles, das im nächsten gesendeten Gamestate erkannt werden kann (siehe \ref{gameend}).
\begin{description}
\item[MSG] Fehlermeldung
\end{description}
\begin{verbatim}
<room roomId="RID">
	<error message="MSG">
		<originalRequest>
		Request, der den Fehler verursacht hat
		</originalRequest>
	</error>
</room>
\end{verbatim}

\section{Spiel pausiert}
Ein \glqq spielfähiger\grqq\ Client muss Pausierungsnachrichten nicht beachten,
da er nur auf Aufforderungen (Zug-Aufforderung siehe \ref{moverequest}) des Servers handelt.
\begin{description}
\item[N] Spielername
\item[P] Punktekonto
\item[card] wie in \ref{card} definiert
\end{description}
\begin{verbatim}
<room roomId="RID">
	<data class="paused">
		<nextPlayer class="player" displayName="N" color="C" points="P">
			<cards>
			  {card}
			</cards>
		</nextPlayer>
	</data>
</room>
\end{verbatim}

\section{Spiel verlassen}
\begin{verbatim}
<left roomId="RID"/>
\end{verbatim}


\section{\label{gameend}Spielergebnis}
Zum Spielende enthält der Spielstatus eine \textit{Condition}, der das Spielergebnis entnommen werden kann:
\begin{description}
\item[W] Gewinner (RED/BLUE), bei Unentschieden wird kein winner mitgeschickt.
\item[R] Text, der den Grund für das Spielende erklärt
\end{description}
\begin{verbatim}
<condition winner="W" reason="R"/>
\end{verbatim}

\newpage
\part{Überblick}
Hier nochmal ein kurzer Überblick, der etwas genauer zeigt, wie die Kommunikation ablaufen kann.\bigskip\\

\begin{enumerate}
\item Ein Spielserver startet ein Spiel und wartet auf den Client (die Clients).
\item Der Client stellt eine TCP Verbindung zum Spielserver her (er kennt dessen IP-Adresse und Port)
\item Die Verbindung ist aufgebaut und der Client sendet \begin{verbatim}
<protocol>
    <join gameType="swc_2013_cartagena"/>
\end{verbatim}
\item Der Server sendet \begin{verbatim}
<protocol>
    <joined roomId="65d3d704-87d9-42eb-8995-51c3e6324513"/>
\end{verbatim}
\item der Client merkt sich die roomId.
\item der Server startet das Spiel und sendet \begin{verbatim}
<room roomId="65d3d704-87d9-42eb-8995-51c3e6324513">
    <data class="welcome" color="BLUE"/>
</room>
\end{verbatim}
\item der Client merkt sich seine Spielerfarbe.
\item der Server sendet das Ausgangsspielfeld: 
\begin{verbatim}
<room roomId="65d3d704-87d9-42eb-8995-51c3e6324513">
  <data class="memento">
    <state class="state" turn="0" currentPlayer="RED">
      <red displayName="Spieler 1" color="RED" points="0">
        <cards>
          <card symbol="BOTTLE"/>
          <card symbol="DAGGER"/>
          <card symbol="DAGGER"/>
          <card symbol="HAT"/>
          <card symbol="KEY"/>
          <card symbol="PISTOL"/>
        </cards>
      </red>
      <blue displayName="Spieler 2" color="BLUE" points="0">
        <cards>
          <card symbol="PISTOL"/>
          <card symbol="SKULL"/>
          <card symbol="KEY"/>
          <card symbol="KEY"/>
          <card symbol="KEY"/>
          <card symbol="DAGGER"/>
        </cards>
      </blue>
      <openCards>
        <card symbol="BOTTLE"/>
        <card symbol="BOTTLE"/>
        <card symbol="HAT"/>
        <card symbol="KEY"/>
        <card symbol="PISTOL"/>
        <card symbol="PISTOL"/>
        <card symbol="PISTOL"/>
        <card symbol="HAT"/>
        <card symbol="PISTOL"/>
        <card symbol="BOTTLE"/>
        <card symbol="KEY"/>
        <card symbol="BOTTLE"/>
      </openCards>
      <board>
        <fields>
          <field type="START">
            <pirates>
              <pirate owner="RED"/>
              <pirate owner="BLUE"/>
              <pirate owner="RED"/>
              <pirate owner="BLUE"/>
              <pirate owner="RED"/>
              <pirate owner="BLUE"/>
              <pirate owner="RED"/>
              <pirate owner="BLUE"/>
              <pirate owner="RED"/>
              <pirate owner="BLUE"/>
              <pirate owner="RED"/>
              <pirate owner="BLUE"/>
            </pirates>
          </field>
          <field type="SYMBOL" symbol="SKULL">
            <pirates/>
          </field>
          <field type="SYMBOL" symbol="DAGGER">
            <pirates/>
          </field>
          <field type="SYMBOL" symbol="KEY">
            <pirates/>
          </field>
          <field type="SYMBOL" symbol="HAT">
            <pirates/>
          </field>
          <field type="SYMBOL" symbol="BOTTLE">
            <pirates/>
          </field>
          <field type="SYMBOL" symbol="PISTOL">
            <pirates/>
          </field>
          <field type="SYMBOL" symbol="HAT">
            <pirates/>
          </field>
          <field type="SYMBOL" symbol="BOTTLE">
            <pirates/>
          </field>
          <field type="SYMBOL" symbol="DAGGER">
            <pirates/>
          </field>
          <field type="SYMBOL" symbol="SKULL">
            <pirates/>
          </field>
          <field type="SYMBOL" symbol="PISTOL">
            <pirates/>
          </field>
          <field type="SYMBOL" symbol="KEY">
            <pirates/>
          </field>
          <field type="SYMBOL" symbol="KEY">
            <pirates/>
          </field>
          <field type="SYMBOL" symbol="HAT">
            <pirates/>
          </field>
          <field type="SYMBOL" symbol="PISTOL">
            <pirates/>
          </field>
          <field type="SYMBOL" symbol="DAGGER">
            <pirates/>
          </field>
          <field type="SYMBOL" symbol="BOTTLE">
            <pirates/>
          </field>
          <field type="SYMBOL" symbol="SKULL">
            <pirates/>
          </field>
          <field type="SYMBOL" symbol="SKULL">
            <pirates/>
          </field>
          <field type="SYMBOL" symbol="HAT">
            <pirates/>
          </field>
          <field type="SYMBOL" symbol="KEY">
            <pirates/>
          </field>
          <field type="SYMBOL" symbol="DAGGER">
            <pirates/>
          </field>
          <field type="SYMBOL" symbol="BOTTLE">
            <pirates/>
          </field>
          <field type="SYMBOL" symbol="PISTOL">
            <pirates/>
          </field>
          <field type="SYMBOL" symbol="BOTTLE">
            <pirates/>
          </field>
          <field type="SYMBOL" symbol="HAT">
            <pirates/>
          </field>
          <field type="SYMBOL" symbol="PISTOL">
            <pirates/>
          </field>
          <field type="SYMBOL" symbol="SKULL">
            <pirates/>
          </field>
          <field type="SYMBOL" symbol="DAGGER">
            <pirates/>
          </field>
          <field type="SYMBOL" symbol="KEY">
            <pirates/>
          </field>
          <field type="FINISH">
            <pirates/>
          </field>
        </fields>
      </board>
    </state>
  </data>
</room>
\end{verbatim}
\item da der andere Spieler anfängt folgt noch eine \textit{memento}-Nachricht vom Server, die den Spielstatus nach dem ersten Zug von Spieler 1 beinhaltet.
\item es folgt die erste Zug-Aufforderung vom Server: \begin{verbatim}
<room roomId="65d3d704-87d9-42eb-8995-51c3e6324513">
    <data class="sc.framework.plugins.protocol.MoveRequest"/>
</room>
\end{verbatim}
\item Der Client antwortet mit einem Zug: \begin{verbatim}
<room roomId="65d3d704-87d9-42eb-8995-51c3e6324513">
    <data class="moveContainer">
      <firstMove class="forwardMove" fieldIndex="0" symbol="DAGGER"/>
      <secondMove class="forwardMove" fieldIndex="2" symbol="KEY"/>
      <thirdMove class="forwardMove" fieldIndex="0" symbol="HAT"/>
  </data>
</room>
\end{verbatim}
\item so geht es weiter...
\item die letzte Nachricht enthält ein \begin{verbatim}
</protocol>
\end{verbatim}
Daraufhin wird die Verbindung geschlossen
\end{enumerate}

\end{document}


<room roomId="RID">
  <data class="result">
    <definition>
      <fragment name="Punkte">
        <aggregation>AVERAGE</aggregation>
        <relevantForRanking>true</relevantForRanking>
      </fragment>
    </definition>
    <score cause="REGULAR">
      <part>12</part>
    </score>
    <score cause="REGULAR">
      <part>8</part>
    </score>
  </data>
</room>