%inhaltszähler
\newcounter{content}[chapter]
\newcounter{subcontent}[content]

%inhaltstyp
\newcommand{\contenttype}[2]{%
\stepcounter{content}%
\subsection*{#2 \thechapter.\thecontent: #1}%
\addcontentsline{toc}{subsection}{\numberline{}#2 \thechapter.\thecontent: #1}}

% inhalte
\newcommand{\definition}[1]{\contenttype{#1}{Definition}}
\newcommand{\proposition}[1]{\contenttype{#1}{Proposition}}
\newcommand{\lemma}[1]{\contenttype{#1}{Lemma}}
\newcommand{\korollar}[1]{\contenttype{#1}{Korollar}}
\newcommand{\satz}[1]{\contenttype{#1}{Satz}}
\newcommand{\folgerung}[1]{\contenttype{#1}{Folgerung}}
\newcommand{\beispiel}[1]{\contenttype{#1}{Beispiel}}


%unterinhaltstyp
\newcommand{\subcontenttype}[2]{%
\stepcounter{subcontent}%
\subsection*{#2 \thechapter.\thecontent.\alph{subcontent}: #1}%
\addcontentsline{toc}{subsection}{\numberline{}#2 \thechapter.\thecontent.\alph{subcontent}: #1}}

%subinhalte
\newcommand{\subdefinition}[1]{\subcontenttype{#1}{Definition}}
\newcommand{\subproposition}[1]{\subcontenttype{#1}{Proposition}}
\newcommand{\sublemma}[1]{\subcontenttype{#1}{Lemma}}
\newcommand{\subkorollar}[1]{\subcontenttype{#1}{Korollar}}
\newcommand{\subsatz}[1]{\subcontenttype{#1}{Satz}}
\newcommand{\subfolgerung}[1]{\subcontenttype{#1}{Folgerung}}
\newcommand{\subbeispiel}[1]{\subcontenttype{#1}{Beispiel}}


%weitere strukturen
\newcommand{\bsp}[1]{\subsubsection*{Beispiel: #1}}
\newcommand{\bsps}[1]{\subsubsection*{Beispiele: #1}}
\newcommand{\beweis}{\subsubsection*{Beweis:}}
\newcommand{\bemerkung}{\subsubsection*{Bemerkung:}}


%für aufgabenzettel und lösungen
\newcounter{series}
\newcounter{task}[series]

\newcommand{\series}{
\stepcounter{series}%
\section*{Serie \theseries}%
\addcontentsline{toc}{section}{\numberline{}{Serie \theseries}}}

\newcommand{\task}{\stepcounter{task}\subsection*{Aufgabe \thetask}\vspace{-2ex}\markright{Aufgabe \thetask}}
\newcommand{\solution}{\subsubsection*{Beweis:}}

