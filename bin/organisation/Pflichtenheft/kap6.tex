\section{Produktfunktionen}

\subsection{Anwendungsfälle aus Benutzersicht}
\label{sec:usecaseuser}

Nachfolgend werden die folgenden Anwendungsfälle detailliert beschrieben: 
\begin{itemize}
	\item Eine eigene, lokale Datei veröffentlichen (Alternativ: mehrere Dateien)
	\item Alle veröffentlichten Daten eines \gls{peer}s anzeigen (Alternativ: alle Peers)
	\item Eine Backup-Datei verteilen (Alternativ: mehrere Backup-Dateien)
	\item Eine veröffentlichte Datei eines \gls{peer}s herunterladen (Alternativ: Backup mit Passwort)
	\item Speicherplatz für andere freigeben
	\item Identifikation mit Benutzername und Passwort
	\item Eine eigene Backup-Datei aus dem \gls{btn} löschen
	\item Eine veröffentlichte Datei anhand von Stichwörtern suchen
\end{itemize}

\subsubsection{Veröffentlichen von lokalen Dateien}
\begin{center}
	\begin{tabular}{|l|p{10cm}|}
		\hline
		Use Case Nummer & V-\useusercounter \\
		\hline
		Use Case Name & Eigene, lokale Datei veröffentlichen \\
		\hline
		Initiierender Akteur & Benutzer \\
		\hline
		Weitere Akteure &  - \\
		\hline
		Kurzbeschreibung & Der Benutzer wählt eine eigene, lokale Datei aus und markiert sie als „ver\-öffentlicht“.\\
		\hline
		Vorbedingung & Die \gls{bta} ist mit dem \gls{btn} verbunden. \\
		\hline
		Nachbedingung & Die Datei wurde veröffentlicht und kann von allen \gls{peer}s angesehen werden. \\
		\hline
		Funktionalität des Use Cases &
		Ablauf:
		\begin{enumerate}
			\item Der Benutzer wählt die eigene, lokale Datei aus
			\item Der Benutzer markiert die Datei als „veröffentlicht“
		\end{enumerate}\\
		\hline
		Alternativen &
		\begin{description}
			\item zu 1) Der Benutzer wählt mehrere Dateien/Ordner aus
			\item zu 2) Abbruch der Auswahl
		\end{description}\\
		\hline
		Ausnahmen & - \\
		\hline
		Benutzte Use Cases & - \\
		\hline
	\end{tabular}
\end{center}

\newpage
 
\subsubsection{Alle veröffentlichten Daten eines Peers anzeigen}
\firstPartUsecaseTable{V-\useusercounter}
{Alle Veröffentlichten Daten eines \gls{peer}s anzeigen}
{Benutzer}
{-}
{Der Benutzer lässt sich alle veröffentlichten Daten eines \gls{peer}s anzeigen}
{Die \gls{bta} ist mit dem \gls{btn} verbunden.}
{Die \gls{bta} zeigt alle Daten eines ausgewählten \gls{peer}s an.}
{Ablauf:\begin{enumerate}\item Der Benutzer wählt einen \gls{peer} aus der aktuellen \gls{peerliste} aus.\item Der Benutzer wählt „Alle veröffentlichten Daten anzeigen“ aus.\item Die \gls{bta} zeigt alle Daten des ausgewählten \gls{peer}s an.\end{enumerate}}
{\begin{description}
	\item zu 1) Der Benutzer wählt mehrere Dateien/Ordner aus.
	\item zu 2) Abbruch der Auswahl.
\end{description}}
Ausnahmen & -\\
\hline
 Benutzte Use Cases & -\\
\hline
\end{tabular}
\end{center}

\newpage

\subsubsection{Eine Backup-Datei verteilen}
\firstPartUsecaseTable{V-\useusercounter}
{Eine Backup-Datei verteilen}
{Benutzer}
{-}
{Der Benutzer wählt eine eigene, lokale Datei aus und markiert sie als „gesichert“.}
{Die \gls{bta} ist mit dem \gls{btn} verbunden.}
{Die Datei wurde gesichert und kann nur von dem Benutzer angesehen werden.}
{Ablauf:\begin{enumerate}\item Der Benutzer wählt die eigene, lokale Datei aus.
\item Der Benutzer markiert die Datei als „gesichert“.\end{enumerate}}
{\begin{description}
	\item zu 1) Der Benutzer wählt mehrere Dateien/Ordner aus.
	\item zu 2) Abbruch der Auswahl.
\end{description}}
Ausnahmen & -\\
\hline
 Benutzte Use Cases & -\\
\hline
\end{tabular}
\end{center}

\newpage

\subsubsection{Eine veröffentlichte Datei herunterladen}
\firstPartUsecaseTable{V-\useusercounter}
{Eine veröffentlichte Datei herunterladen}
{Benutzer}
{-}
{Der Benutzer lädt eine veröffentlichte Datei eines Peers herunter.}
{Die \gls{bta} ist mit dem \gls{btn} verbunden.
Die Liste der öffentlichen Dateien eines \gls{peer}s wird angezeigt.}
{Die \gls{bta} zeigt die heruntergeladene Datei an.}
{Ablauf:\begin{enumerate} \item Der Benutzer wählt eine veröffentlichte Datei eines \gls{peer}s aus der Liste.
\item Der Benutzer lädt diese Datei herunter.\end{enumerate}}
{\begin{description} \item zu 1) Der Benutzer lädt eine von ihm gesicherte Datei herunter, nachdem das richtige  Passwort eingegeben wurde.\end{description}}
Ausnahmen & -\\
\hline
 Benutzte Use Cases & -\\
\hline
\end{tabular}
\end{center}

\newpage

\subsubsection{Speicherplatz für andere freigeben}
\firstPartUsecaseTable{V-\useusercounter}
{Speicherplatz für andere freigeben}
{Benutzer}
{-}
{Der Benutzer wählt einen eigenen, lokalen Ordner und stellt die maximale Größe ein.}
{}
{Ein eigener Ordner ist für die empfangenen Backup-Dateien anderer mit einer maximalen Größe bereit.}
{Ablauf:\begin{enumerate}\item Der Benutzer wählt den eigenen, lokalen Ordner aus.
\item Der Benutzer legt eine maximale Größe fest.\end{enumerate}}
{\begin{description}
	\item zu 2) Abbruch der Auswahl.
\end{description}}
Ausnahmen & -\\
\hline
 Benutzte Use Cases & -\\
\hline
\end{tabular}
\end{center}

\newpage

\subsubsection{Identifikation mit Benutzername und Passwort}
\firstPartUsecaseTable{V-\useusercounter}
{Identifikation mit Benutzername und Passwort}
{Benutzer}
{-}
{Der Benutzer identifiziert sich mit einem Benutzernamen und Passwort.}
{Die \gls{bta} ist mit dem \gls{btn} verbunden.}
{Der Benutzer ist identifiziert.}
{Ablauf: \begin{enumerate} \item Der Benutzer gibt Benutzernamen und Passwort ein.\end{enumerate}}
{\begin{description} \item zu 1) Es erfolgt keine Eingabe seitens des Benutzers.\end{description}}
Ausnahmen & -\\
\hline
 Benutzte Use Cases & -\\
\hline
\end{tabular}
\end{center}

\newpage

\subsubsection{Backup-Datei löschen}
\firstPartUsecaseTable{V-\useusercounter}
{Eigene Backup-Datei löschen}
{Benutzer}
{-}
{Der Benutzer löscht eine seiner Backup-Dateien aus dem \gls{btn}.}
{Die \gls{bta} ist mit dem \gls{btn} verbunden.}
{Die eigene Backup-Datei wurde aus dem \gls{btn} gelöscht.}
{Ablauf: \begin{enumerate} \item Der Benutzer lässt sich die Liste seiner Backup-Dateien anzeigen.
\item Der Benutzer markiert eine Datei aus der Liste. \item Der Benutzer klickt auf „Backup-Datei löschen“.\end{enumerate}}
{\begin{description} \item zu 2) Der Benutzer bricht ab.\end{description}}
Ausnahmen & -\\
\hline
 Benutzte Use Cases & -\\
\hline
\end{tabular}
\end{center}

\newpage

\subsubsection{Veröffentlichte Dateien suchen}
\firstPartUsecaseTable{V-\useusercounter}
{Veröffentlichte Dateien anhand von Stichwörtern suchen}
{Benutzer}
{-}
{Der Benutzer gibt Stichwörter ein und sucht damit nach veröffentlichten Dateien.}
{Die \gls{bta} ist mit dem \gls{btn} verbunden.}
{Die Liste der veröffentlichten Dateien, die zu den Stichwörtern passen, wird angezeigt.}
{Ablauf: \begin{enumerate} \item Der Benutzer gibt Stichwörter ein.
\item Die \gls{bta} zeigt alle veröffentlichten Daten mit den eingegebenen Stichwörtern an.\end{enumerate}}
{-}
Ausnahmen & -\\
\hline
 Benutzte Use Cases & -\\
\hline
\end{tabular}
\end{center}

\newpage

%-------------------------------------------------------------------------------------
%-------------------------------------------------------------------------------------
%-------------------------------------------------------------------------------------

\subsection{Anwendungsfälle aus Systemsicht}
\label{sec:usecasesystem}

Nachfolgend werden die folgenden Anwendungsfälle detailliert beschrieben: 
\begin{itemize}
	\item Verbindung herstellen
	\item Verbindung trennen
	\item Nachricht validieren
	\item \gls{peerliste} aktualisieren
	\item Verteilung überprüfen
	\item Daten verschlüsseln
	\item Daten entschlüsseln
	\item Daten komprimieren
	\item Daten dekomprimieren
	\item Daten teilen
	\item Daten zusammenfügen
	\item \gls{backup}-Daten verteilen
\end{itemize}

\subsubsection{Verbindung herstellen}
\firstPartUsecaseTable{S-\usesystemcounter}
{Eine Verbindung mit dem \gls{btn} herstellen}
{System}
{-}
{Das System stellt eine Verbindung zum \gls{btn} her.}
{Es besteht keine Verbindung zum \gls{btn}. Es ist zu mindestens die Adresse von einem \gls{peer} im \gls{btn} bekannt.}
{Es besteht eine Verbindung zum \gls{btn}.}
{Ablauf:
	\begin{enumerate}
		\item Das System verbindet sich mit dem bekannten \gls{peer}.
	\end{enumerate}}
{-}
Ausnahmen 	& Es konnte keine Verbindung hergestellt werden.\\
\hline
 Benutzte Use Cases & -\\
\hline
\end{tabular}
\end{center}

\newpage

\subsubsection{Verbindung trennen}
\firstPartUsecaseTable{S-\usesystemcounter}
{Die Verbindung mit dem \gls{btn} trennen}
{System}
{-}
{Das System trennt die Verbindung zum \gls{btn}.}
{Es besteht eine Verbindung zum \gls{btn}.}
{Es besteht keine Verbindung zum \gls{btn}.}
{Ablauf:
	\begin{enumerate}
		\item Das System sendet, empfängt und antwortet auf keine Nachrichten mehr.
	\end{enumerate}}
{-}
Ausnahmen 	& -\\
\hline
 Benutzte Use Cases & -\\
\hline
\end{tabular}
\end{center}

\newpage

\subsubsection{Nachricht validieren}
\firstPartUsecaseTable{S-\usesystemcounter}
{Eine eingehende Nachricht validieren}
{System}
{-}
{Das System prüft, ob eine eingehende Nachricht im korrekten Format vorliegt.}
{Das System kennt eine Spezifikation, anhand der es die Korrektheit einer Nachricht nachweisen kann.}
{Die Nachricht wurde als korrekt oder nicht korrekt identifiziert.}
{Ablauf:
	\begin{enumerate}
		\item Das System prüft, ob die eingehende \gls{xml}-Nachricht einem vorgegebenen \gls{xml}-Schema entspricht.
		\item Die Nachricht wird als korrekt identifiziert.
	\end{enumerate}}
{\begin{description}
		\item zu 2) die Nachricht wurde als nicht korrekt identifiziert.
	\end{description}}
Ausnahmen 	& -\\
\hline
 Benutzte Use Cases & -\\
\hline
\end{tabular}
\end{center}

\newpage

\subsubsection{Peerliste aktualisieren}
\label{usecase:peerliste}
\firstPartUsecaseTable{S-\usesystemcounter}
{Peerliste aktualisieren}
{System}
{-}
{Das System prüft, welche \gls{node}s in der \gls{peerliste} online sind und welche offline sind.}
{Es besteht eine Verbindung zum \gls{btn}}
{Das System hält eine aktuelle \gls{peerliste}.}
{Ablauf:
	\begin{enumerate}
		\item Jede Stunde verschickt das System seine \gls{peerliste} an alle \gls{peer}s.
		\item Das System kriegt \gls{peerliste}n von jedem \gls{peer} zurück und setzt das Online-Flag dieses \gls{peer}s auf \texttt{true}.
		\item Das System fügt die fremden Informationen in seine \gls{peerliste} ein.
	\end{enumerate}}
{\begin{description}
		\item zu 2) das System kriegt von einem Knoten keine \gls{peerliste} und setzt das Online-Flag des \gls{peer}s in der \gls{peerliste} auf \texttt{false}. 
	\end{description}}
Ausnahmen 	& -\\
\hline
 Benutzte Use Cases & -\\
\hline
\end{tabular}
\end{center}

\newpage

\subsubsection{Verteilung überprüfen}
\firstPartUsecaseTable{S-\usesystemcounter}
{Die Verteilung von Dateiteilen überprüfen}
{System}
{-}
{Das System prüft, ob die Dateiteile ausreichend im Netz verteilt sind.}
{Das System hat Informationen über die Dateiteile und eine aktuelle \gls{peerliste}. Anforderungen an die Anzahl der Verteilung der Dateiteile.}
{Die Verteilung wurde als ausreichend oder nicht ausreichend gekennzeichnet.}
{Ablauf:
	\begin{enumerate}
		\item Das System schickt an alle \gls{peer}s eine Anfrage, ob sie den Dateiteil x haben.
		\item Das System überprüft, ob die Anzahl der Dateiteile x im System den Anforderungen genügt.
		\item Das System bewertet die Verteilung als ausreichend.
	\end{enumerate}}
{\begin{description}
		\item zu 3) das System bewertet die Verteilung als nicht ausreichend.
	\end{description}}
Ausnahmen 	& -\\
\hline
 Benutzte Use Cases & -\\
\hline
\end{tabular}
\end{center}

\newpage

\subsubsection{Daten verschlüsseln}
\firstPartUsecaseTable{S-\usesystemcounter}
{\gls{daten} mit einem Passwort verschlüsseln}
{System}
{-}
{Das System verschlüsselt die \gls{daten} mit einem Passwort}
{\gls{daten}, die verschlüsselt werden soll. Das System besitzt Zugriffsrechte für die \gls{daten}. Es ist genügend Speicherplatz vorhanden.}
{Die \gls{daten} sind verschlüsselt.}
{Ablauf:
	\begin{enumerate}
		\item Das System liest die \gls{daten} ein.
		\item Das System verschlüsselt die \gls{daten} mit Hilfe eines Verschlüsselungsalgorithmus.
		\item Das System legt die verschlüsselten \gls{daten} ab.
	\end{enumerate}}
{-}
Ausnahmen & Es ist nicht genügend Speicherplatz vorhanden.\\
\hline
 Benutzte Use Cases & -\\
\hline
\end{tabular}
\end{center}

\newpage

\subsubsection{Daten entschlüsseln}
\firstPartUsecaseTable{S-\usesystemcounter}
{\gls{daten} mit einem Passwort entschlüsseln}
{System}
{-}
{Das System entschlüsselt die \gls{daten} mit einem Passwort.}
{\gls{daten}, die entschlüsselt werden soll und verschlüsselt sind. Das System besitzt Zugriffsrechte für die \gls{daten}. Es ist genügend Speicherplatz vorhanden. Ein Passwort ist vorhanden.}
{Die \gls{daten} sind entschlüsselt.}
{Ablauf:
	\begin{enumerate}
		\item Das System liest die \gls{daten} ein.
		\item Das System entschlüsselt die \gls{daten} mit Hilfe eines Verschlüsselungsalgorithmus und dem Passwort.
		\item Das System legt die entschlüsselten \gls{daten} ab.
	\end{enumerate}}
{\begin{description}
		\item zu 3) die \gls{daten} wurden nicht erfolgreich entschlüsselt, weil das Passwort falsch war.
	\end{description}}
Ausnahmen & Es ist nicht genügend Speicherplatz vorhanden.\\
\hline
 Benutzte Use Cases & -\\
\hline
\end{tabular}
\end{center}

\newpage

\subsubsection{Daten komprimieren}
\firstPartUsecaseTable{S-\usesystemcounter}
{\gls{daten} komprimieren}
{System}
{-}
{Das System komprimiert die \gls{daten}.}
{\gls{daten}, die komprimiert werden soll. Das System besitzt Zugriffsrechte für die \gls{daten}. Es ist genügend Speicherplatz vorhanden.}
{Die \gls{daten} sind komprimiert.}
{Ablauf:
	\begin{enumerate}
		\item Das System liest die \gls{daten} ein.
		\item Das System komprimiert die \gls{daten} mit Hilfe eines Komprimierungsalgorithmuses.
		\item Das System legt die komprimierten \gls{daten} ab.
	\end{enumerate}}
{-}
Ausnahmen & Es ist nicht genügend Speicherplatz vorhanden.\\
\hline
 Benutzte Use Cases & -\\
\hline
\end{tabular}
\end{center}

\newpage

\subsubsection{Daten dekomprimieren}
\firstPartUsecaseTable{S-\usesystemcounter}
{\gls{daten} dekomprimieren}
{System}
{-}
{Das System dekomprimiert die \gls{daten}.}
{\gls{daten}, die dekomprimiert sind. Das System besitzt Zugriffsrechte für die \gls{daten}. Es ist genügend Speicherplatz vorhanden.}
{Die \gls{daten} sind dekomprimiert.}
{Ablauf:
	\begin{enumerate}
		\item Das System liest die \gls{daten} ein.
		\item Das System dekomprimiert die \gls{daten}.
		\item Das System legt die dekomprimierten \gls{daten} ab.
	\end{enumerate}}
{-}
Ausnahmen & Es ist nicht genügend Speicherplatz vorhanden.\\
\hline
 Benutzte Use Cases & -\\
\hline
\end{tabular}
\end{center}

\newpage

\subsubsection{Daten teilen}
\firstPartUsecaseTable{S-\usesystemcounter}
{\gls{daten} teilen in mehrere kleinere Dateiteile}
{System}
{-}
{Das System teilt die \gls{daten} in mehrere kleinere Dateiteile.}
{\gls{daten}, die geteilt werden sollen. Das System besitzt Zugriffsrechte für die \gls{daten}. Es ist genügend Speicherplatz vorhanden.}
{Die \gls{daten} sind geteilt in mehrere kleinere Dateiteile.}
{Ablauf:
	\begin{enumerate}
		\item Das System komprimiert die \gls{daten} und teilt dabei die \gls{daten} in mehrere kleinere Dateiteile.
		\item Das System legt die Dateiteile ab.
	\end{enumerate}}
{-}
Ausnahmen & Es ist nicht genügend Speicherplatz vorhanden.\\
\hline
 Benutzte Use Cases & -\\
\hline
\end{tabular}
\end{center}

\newpage

\subsubsection{Daten zusammenfügen}
\firstPartUsecaseTable{S-\usesystemcounter}
{Dateiteile zusammenfügen zu den originalen \gls{daten}}
{System}
{-}
{Das System fügt die Dateiteile zu den originalen \gls{daten} zusammen.}
{Dateiteile, die zusammengefügt werden sollen und alle Dateiteile zu den originalen \gls{daten}. Das System besitzt Zugriffsrechte für die Dateiteile. Es ist genügend Speicherplatz vorhanden.}
{Die originalen \gls{daten} stehen wieder zur Verfügung.}
{Ablauf:
	\begin{enumerate}
		\item Das System dekomprimiert die \gls{daten} aus den Dateiteilen.
		\item Das System legt die \gls{daten} ab.
	\end{enumerate}}
{-}
Ausnahmen & Es ist nicht genügend Speicherplatz vorhanden.\\
\hline
 Benutzte Use Cases & -\\
\hline
\end{tabular}
\end{center}

\newpage

\subsubsection{Backup-Daten verteilen}
\firstPartUsecaseTable{S-\usesystemcounter}
{\gls{backup}-\gls{daten} verteilen}
{System}
{\gls{peer}}
{Das System verteilt eine \gls{backup}-Datei redundant an mehrere \gls{peer}s.}
{Es besteht eine Verbindung zum \gls{btn}, die zu verteilenden \gls{daten} sind komprimiert, verschlüsselt und aufgeteilt worden und die zugehörigen Meta-\gls{daten} sind verschlüsselt worden.}
{Die \gls{backup}-\gls{daten} wurden vollständig gemäß der \gls{redqua} verteilt.}
{Ablauf:
	\begin{enumerate}
		\item Das System öffnet einen \gls{tcp}-\gls{port}.
		\item Das System sendet solange sukzessiv Speicher-Anfragen (\ref{sec:speicheranfrage}) (mit dem geöffneten \gls{tcp}-\gls{port}) pro \gls{btfp} von den zu verteilenden \gls{backup}-\gls{daten}  gemäß der \gls{redqua} an verschiedene \gls{peer}s aus der eigenen \gls{peerliste}, bis alle \gls{btfp} erfolgreich versandt wurden.
		\item Jeder \gls{peer}, der einen \gls{btfp} speichern kann, lädt einen solchen über den \gls{tcp}-\gls{port} herunter.
		\item Bei kompletter, erfolgreicher Übertragung schließt das System den geöffneten \gls{tcp}-\gls{port}.
	\end{enumerate}}
{-}
Ausnahmen & zu 2.) Das System kann keine Verbindung zum \gls{peer} herstellen: Das Online-Flag (\ref{usecase:peerliste}) wird auf false gesetzt und die Speicher-Anfrage wird an einen anderer \gls{peer} gesendet.\\
	& zu 2.) Das System konnte nicht alle \gls{btfp} gemäß der \gls{redqua} versenden: Das System benachrichtigt Benutzer.\\
\hline
 Benutzte Use Cases & -\\
\hline
\end{tabular}
\end{center}
