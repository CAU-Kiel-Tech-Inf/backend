\section{Produktdaten}
Es werden nachfolgend alle Daten und Datenstrukturen angegeben, die die \gls{bta} speichert und benutzt:

\subsection{Peerliste und Nodes}
Die \gls{bta} hält eine Liste (die sogenannte \gls{peerliste}) mit allen \gls{node}s im \gls{btn}, wobei jede \gls{node} mit folgenden Informationen gespeichert wird:
\begin{itemize}
	\item \gls{ip}-Adresse
	\item \gls{udp}-\gls{port}
	\item Zeitstempel der letzten Aktivität des \gls{node}s
	\item Gesamter Speicherplatz
	\item Freier Speicherplatz
	\item Online-Flag \ref{usecase:peerliste}
\end{itemize}

{\Large Die empfangenen Dateiteile werden lokal mit einem zusätzlichen Informationsteil abgelegt, der wie folgt aussieht}:

\subsection{Veröffentlichte Dateien}
\begin{itemize}
	\item \gls{hash} \"uber die Meta-Daten und die Daten
	\item Meta-Daten:
	\begin{itemize}
		\item Titel
		\item Schl\"usselw\"orter
		\item Benutzername des Autors
		\item Teil x von y
		\item Datum und Uhrzeit der letzten \"Anderung
		\item Gesamt-\gls{hash}
	\end{itemize}
\end{itemize}

\newpage

\subsection{Sicherungsdateien}
\begin{itemize}
	\item Benutzername des Autors
	\item Benutzername des Autors (verschl\"usselt)
	\item \gls{hash} \"uber die verschl\"usselten Meta-Daten und die verschl\"usselten Daten
	\item Meta-Daten (verschl\"usselt):
		\begin{itemize}
			\item Titel
			\item Teil x von y
			\item Datum und Uhrzeit der letzten \"Anderung
			\item Gesamt-\gls{hash}
			\item \gls{Part-Passwort}
		\end{itemize}
\end{itemize}
