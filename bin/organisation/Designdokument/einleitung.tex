\section{Einleitung}
Dieses Dokument stellt eine Dokumentation bezüglich des Designs von Software Challenge dar. Es richtet sich
in erster Linie an Personen, von denen weiterentwickelt werden soll, da hier das Verständnis des
bestehenden Designs essenzieller Natur ist.
Es existiert bereits ein Pflichtenheft, welches die grundlegenden Features des zu erstellenden Systems
zusammenfasst. Des Weiteren sind verschiedenste Use Cases notiert, welche einen genaueren
Überblick über die geplanten Programmabläufe geben.

\subsection{Verweise}
Um eine möglichst einfache Entwicklung insgesamt zu gewährleisten, wurde Eclipse als IDE verwendet.
Eclipse1 ist eine kostenfreie Plattform, welche nativ die Sprache Java unterstützt.
Da in Java 1.5 realisiert wurde, wird angenommen, dass der Leser vertraut im Umgang mit Java
allgemein und im speziellen mit den Features der Version 1.5. Java ist eine freie Programmiersprache,
welche von Sun entwickelt wird. Java und weitere Informationen können von der Internetpräsenz von
Sun2 bezogen werden.
Zur Erstellung und Visualisierung des Designs wurde Omondo3 verwendet. Omondo ist ein Plug-In für
Eclipse, welches auf die Erstellung und Darstellung von UML-nahen Inhalten spezialisiert ist.
In dem häufig referenzierten Buch „Design Patterns: Elements of Reusable Object-Oriented Software“
von den Autoren Erich Gamma, Richard Helm, Ralph Johnson und John Vlissides werden alle
verwendeten Patterns ausführlich erläutert.
Bezüglich der Verschlüsselungs- und Signierungsalgorithmen baut auf eine Crypto-Bibliothek namens
„FlexiProvider“ auf. „FlexiProvider“ ist ein Produkt des Fachgebiets Theoretische Informatik an der
Technischen Universität Darmstadt. Ausführlichere Informationen über diese Bibliothek können auf der
Internetseite4 eingesehen werden. Da FlexiProvider ausschließlich eine Implementierung der von der
JCA/JCE (Java Cryptographic Architecture/Extension) definierten Schnittstellen ist, lohnt ein Blick auf
die Entsprechenden Ausschnitte der umfangreichen Dokumentation5 von Sun.
Um die Verwendung der PKCSs (Public Key Cryptographic Standards) zu gewährleisten wird ein
Codec Package des Fraunhofer Instituts für graphische Datenverarbeitung verwendet. Dieses kann
ebenfalls von 1 bezogen werden.
Im Zuge der Sicherung der sensiblen Daten wird auf den Java KeyStore6 zugegriffen. Dies ist ein
System, welches auf die sichere und persistente Speicherung von Schlüsseln und Zertifikaten
spezialisiert ist.
1

\subsection{Überblick}
Kapitel 2 wird einen kurzen Abriss über die Gesamtstruktur des Systems geben. Hier wird das
bestehende Design in seiner Gesamtheit und sehr abstrakt analysiert. In Kapitel 3 wird genauer auf
das Design eingegangen. Die wichtigsten Abstraktionsebenen und Designentscheidungen werden hier
ausführlich behandelt. Dies bedeutet, dass auch Alternativen vorgestellt und kritische betrachtet
werden. In Kapitel 4 folgt eine Kurzbeschreibung der eingesetzten Patterns. Kapitel 5 rundet dieses
Dokument mit einer konkreten Analyse der Mechanismen ab, welche das Design von Crypt2Go
ausmachen.
